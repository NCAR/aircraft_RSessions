\batchmode
\makeatletter
\def\input@path{{/home/cooperw/RStudio/RSessions/Session3//}}
\makeatother
\documentclass{beamer}\usepackage[]{graphicx}\usepackage[]{color}
%% maxwidth is the original width if it is less than linewidth
%% otherwise use linewidth (to make sure the graphics do not exceed the margin)
\makeatletter
\def\maxwidth{ %
  \ifdim\Gin@nat@width>\linewidth
    \linewidth
  \else
    \Gin@nat@width
  \fi
}
\makeatother

\definecolor{fgcolor}{rgb}{0.345, 0.345, 0.345}
\newcommand{\hlnum}[1]{\textcolor[rgb]{0.686,0.059,0.569}{#1}}%
\newcommand{\hlstr}[1]{\textcolor[rgb]{0.192,0.494,0.8}{#1}}%
\newcommand{\hlcom}[1]{\textcolor[rgb]{0.678,0.584,0.686}{\textit{#1}}}%
\newcommand{\hlopt}[1]{\textcolor[rgb]{0,0,0}{#1}}%
\newcommand{\hlstd}[1]{\textcolor[rgb]{0.345,0.345,0.345}{#1}}%
\newcommand{\hlkwa}[1]{\textcolor[rgb]{0.161,0.373,0.58}{\textbf{#1}}}%
\newcommand{\hlkwb}[1]{\textcolor[rgb]{0.69,0.353,0.396}{#1}}%
\newcommand{\hlkwc}[1]{\textcolor[rgb]{0.333,0.667,0.333}{#1}}%
\newcommand{\hlkwd}[1]{\textcolor[rgb]{0.737,0.353,0.396}{\textbf{#1}}}%

\usepackage{framed}
\makeatletter
\newenvironment{kframe}{%
 \def\at@end@of@kframe{}%
 \ifinner\ifhmode%
  \def\at@end@of@kframe{\end{minipage}}%
  \begin{minipage}{\columnwidth}%
 \fi\fi%
 \def\FrameCommand##1{\hskip\@totalleftmargin \hskip-\fboxsep
 \colorbox{shadecolor}{##1}\hskip-\fboxsep
     % There is no \\@totalrightmargin, so:
     \hskip-\linewidth \hskip-\@totalleftmargin \hskip\columnwidth}%
 \MakeFramed {\advance\hsize-\width
   \@totalleftmargin\z@ \linewidth\hsize
   \@setminipage}}%
 {\par\unskip\endMakeFramed%
 \at@end@of@kframe}
\makeatother

\definecolor{shadecolor}{rgb}{.97, .97, .97}
\definecolor{messagecolor}{rgb}{0, 0, 0}
\definecolor{warningcolor}{rgb}{1, 0, 1}
\definecolor{errorcolor}{rgb}{1, 0, 0}
\newenvironment{knitrout}{}{} % an empty environment to be redefined in TeX

\usepackage{alltt}
\usepackage{mathptmx}
\usepackage[T1]{fontenc}
\usepackage[latin9]{inputenc}
\setlength{\parskip}{\medskipamount}
\setlength{\parindent}{0pt}
\usepackage{pifont}
\usepackage{amsmath}
\usepackage{amssymb}

\makeatletter
%%%%%%%%%%%%%%%%%%%%%%%%%%%%%% Textclass specific LaTeX commands.
 % this default might be overridden by plain title style
 \newcommand\makebeamertitle{\frame{\maketitle}}%
 % (ERT) argument for the TOC
 \AtBeginDocument{%
   \let\origtableofcontents=\tableofcontents
   \def\tableofcontents{\@ifnextchar[{\origtableofcontents}{\gobbletableofcontents}}
   \def\gobbletableofcontents#1{\origtableofcontents}
 }

%%%%%%%%%%%%%%%%%%%%%%%%%%%%%% User specified LaTeX commands.
\usetheme{WAC}
\setbeamertemplate{headline}{}
\setbeamertemplate{footline}[default]{}
\setbeamertemplate{navigation symbols}{}
%\setbeamercovered{transparent}
%\setbeamercovered{opaque}
\setbeamertemplate{enumerate subitem}{(\alph{enumii})}

\AtBeginDocument{
  \def\labelitemi{\ding{227}}
}

\makeatother
\IfFileExists{upquote.sty}{\usepackage{upquote}}{}
\begin{document}
\setbeamercolor{normal text}{bg=yellow!10}


\title{Session 3: Basics of R}


\subtitle{math operations; using variables}


\author{Al Cooper }


\date{RAF Sessions on R and RStudio}

\makebeamertitle


%\beamerdefaultoverlayspecification{<+->}
\begin{frame}{R as a Calculator}

\begin{block}{}
{Calculator-like operations}
\begin{itemize}
\item Standard interactive R
\item RStudio console provides some conveniences
\item Can do some simple programming interactively
\end{itemize}
\end{block}
\begin{exampleblock}{}
{Example: Roll angle for a 4-min turn}

\[
\frac{v^{2}}{r}=g\thinspace\tan\phi
\]
\[
2\pi r=vT
\]
\[
\phi=\arctan\left(\frac{2\pi v}{gT}\right)
\]



\end{exampleblock}
\end{frame}
\begin{frame}[fragile]{In RStudio Console:}

\begin{exampleblock}{}


\[
\phi=\arctan\left(\frac{2\pi v}{gT}\right)
\]



\begin{knitrout}
\definecolor{shadecolor}{rgb}{0.969, 0.969, 0.969}\color{fgcolor}\begin{kframe}
\begin{alltt}
\hlstd{TAS} \hlkwb{<-} \hlnum{200}
\hlstd{gravity} \hlkwb{<-} \hlnum{9.8}
\hlstd{(}\hlnum{180} \hlopt{/} \hlstd{pi)} \hlopt{*} \hlkwd{atan} \hlstd{(}\hlnum{2} \hlopt{*} \hlstd{pi} \hlopt{*} \hlstd{TAS} \hlopt{/} \hlstd{(gravity} \hlopt{*} \hlnum{240}\hlstd{))}
\end{alltt}
\begin{verbatim}
## [1] 28.11
\end{verbatim}
\end{kframe}
\end{knitrout}


\framesubtitle{focus on what might seem different}
\begin{columns}


\column{5.5cm}
\begin{block}{Operator precedence:}

\begin{itemize}
\item :: \$ {[} {]} PEU:MDAS\\
1:10{*}2 the : has precedence

\begin{itemize}
\item ! (\& \&\&) (| ||) xor
\item <- or = have lowest precedence
\item note \& higher than |
\end{itemize}
\end{itemize}

\column{5.5cm}
\begin{exampleblock}{Operators to note:}


exponentiation: \textasciicircum{} (accepts {*}{*})


modulus: \%\%


integer division: \%/\%


define vector: c(...)


test if element present: \%in\%


matrix multiplication: \%x\%

\end{exampleblock}
\end{block}
\end{columns}

\end{exampleblock}
\end{frame}
\begin{frame}
\end{frame}

\end{document}
